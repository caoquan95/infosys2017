\documentclass{article}

\usepackage{fancyhdr} % Required for custom headers
\usepackage{lastpage} % Required to determine the last page for the footer
\usepackage{amsmath}
\usepackage{extramarks} % Required for headers and footers
\usepackage[usenames,dvipsnames]{color} % Required for custom colors
\usepackage{graphicx} % Required to insert images
\usepackage{listings} % Required for insertion of code
\usepackage{courier} % Required for the courier font
\usepackage{lipsum} % Used for inserting dummy 'Lorem ipsum' text into the template

% Margins
\topmargin=-0.45in
\evensidemargin=0in
\oddsidemargin=0in
\textwidth=6.5in
\textheight=9.0in
\headsep=0.25in

\linespread{1.1} % Line spacing

% Set up the header and footer
\pagestyle{fancy}
\lhead{\hmwkAuthorName} % Top left header
\chead{\hmwkClass\ (\hmwkClassInstructor\ \hmwkClassTime): \hmwkTitle} % Top center head
\rhead{\firstxmark} % Top right header
\lfoot{\lastxmark} % Bottom left footer
\cfoot{} % Bottom center footer
\rfoot{Page\ \thepage\ of\ \protect\pageref{LastPage}} % Bottom right footer
\renewcommand\headrulewidth{0.4pt} % Size of the header rule
\renewcommand\footrulewidth{0.4pt} % Size of the footer rule

\setlength\parindent{0pt} % Removes all indentation from paragraphs

%----------------------------------------------------------------------------------------
%	CODE INCLUSION CONFIGURATION
%----------------------------------------------------------------------------------------

\definecolor{MyDarkGreen}{rgb}{0.0,0.4,0.0} % This is the color used for comments
\lstloadlanguages{Perl} % Load Perl syntax for listings, for a list of other languages supported see: ftp://ftp.tex.ac.uk/tex-archive/macros/latex/contrib/listings/listings.pdf
\lstset{language=Perl, % Use Perl in this example
        frame=single, % Single frame around code
        basicstyle=\small\ttfamily, % Use small true type font
        keywordstyle=[1]\color{Blue}\bf, % Perl functions bold and blue
        keywordstyle=[2]\color{Purple}, % Perl function arguments purple
        keywordstyle=[3]\color{Blue}\underbar, % Custom functions underlined and blue
        identifierstyle=, % Nothing special about identifiers                                         
        commentstyle=\usefont{T1}{pcr}{m}{sl}\color{MyDarkGreen}\small, % Comments small dark green courier font
        stringstyle=\color{Purple}, % Strings are purple
        showstringspaces=false, % Don't put marks in string spaces
        tabsize=5, % 5 spaces per tab
        %
        % Put standard Perl functions not included in the default language here
        morekeywords={rand},
        %
        % Put Perl function parameters here
        morekeywords=[2]{on, off, interp},
        %
        % Put user defined functions here
        morekeywords=[3]{test},
       	%
        morecomment=[l][\color{Blue}]{...}, % Line continuation (...) like blue comment
        numbers=left, % Line numbers on left
        firstnumber=1, % Line numbers start with line 1
        numberstyle=\tiny\color{Blue}, % Line numbers are blue and small
        stepnumber=5 % Line numbers go in steps of 5
}

% Creates a new command to include a perl script, the first parameter is the filename of the script (without .pl), the second parameter is the caption
\newcommand{\perlscript}[2]{
\begin{itemize}
\item[]\lstinputlisting[caption=#2,label=#1]{#1.pl}
\end{itemize}
}

%----------------------------------------------------------------------------------------
%	DOCUMENT STRUCTURE COMMANDS
%	Skip this unless you know what you're doing
%----------------------------------------------------------------------------------------

% Header and footer for when a page split occurs within a problem environment
\newcommand{\enterProblemHeader}[1]{
\nobreak\extramarks{#1}{#1 continued on next page\ldots}\nobreak
\nobreak\extramarks{#1 (continued)}{#1 continued on next page\ldots}\nobreak
}

% Header and footer for when a page split occurs between problem environments
\newcommand{\exitProblemHeader}[1]{
\nobreak\extramarks{#1 (continued)}{#1 continued on next page\ldots}\nobreak
\nobreak\extramarks{#1}{}\nobreak
}

\setcounter{secnumdepth}{0} % Removes default section numbers
\newcounter{homeworkProblemCounter} % Creates a counter to keep track of the number of problems

\newcommand{\homeworkProblemName}{}
\newenvironment{homeworkProblem}[1][Problem \arabic{homeworkProblemCounter}]{ % Makes a new environment called homeworkProblem which takes 1 argument (custom name) but the default is "Problem #"
\stepcounter{homeworkProblemCounter} % Increase counter for number of problems
\renewcommand{\homeworkProblemName}{#1} % Assign \homeworkProblemName the name of the problem
\section{\homeworkProblemName} % Make a section in the document with the custom problem count
\enterProblemHeader{\homeworkProblemName} % Header and footer within the environment
}{
\exitProblemHeader{\homeworkProblemName} % Header and footer after the environment
}

\newcommand{\problemAnswer}[1]{ % Defines the problem answer command with the content as the only argument
\noindent\framebox[\columnwidth][c]{\begin{minipage}{0.98\columnwidth}#1\end{minipage}} % Makes the box around the problem answer and puts the content inside
}

\newcommand{\homeworkSectionName}{}
\newenvironment{homeworkSection}[1]{ % New environment for sections within homework problems, takes 1 argument - the name of the section
\renewcommand{\homeworkSectionName}{#1} % Assign \homeworkSectionName to the name of the section from the environment argument
\subsection{\homeworkSectionName} % Make a subsection with the custom name of the subsection
\enterProblemHeader{\homeworkProblemName\ [\homeworkSectionName]} % Header and footer within the environment
}{
\enterProblemHeader{\homeworkProblemName} % Header and footer after the environment
}

%----------------------------------------------------------------------------------------
%	NAME AND CLASS SECTION
%----------------------------------------------------------------------------------------

\newcommand{\hmwkTitle}{Report \#7} % Assignment title
\newcommand{\hmwkDueDate}{} % Due date
\newcommand{\hmwkClass}{MI\ 1.02} % Course/class
\newcommand{\hmwkClassTime}{5:30pm} % Class/lecture time
\newcommand{\hmwkClassInstructor}{Nghiem Thi Phuong} % Teacher/lecturer
\newcommand{\hmwkAuthorName}{Cao Anh Quan} % Your name

%----------------------------------------------------------------------------------------
%	TITLE PAGE
%----------------------------------------------------------------------------------------

\title{
\vspace{2in}
\textmd{\textbf{\hmwkClass:\ \hmwkTitle}}\\
\normalsize\vspace{0.1in}\small{Due\ on\ \hmwkDueDate}\\
\vspace{0.1in}\large{\textit{\hmwkClassInstructor\ \hmwkClassTime}}
\vspace{3in}
}

\author{\textbf{\hmwkAuthorName}}
\date{} % Insert date here if you want it to appear below your name

%----------------------------------------------------------------------------------------

\begin{document}

\maketitle

%----------------------------------------------------------------------------------------
%	TABLE OF CONTENTS
%----------------------------------------------------------------------------------------

%\setcounter{tocdepth}{1} % Uncomment this line if you don't want subsections listed in the ToC


\newpage

%----------------------------------------------------------------------------------------
%	PROBLEM 1
%----------------------------------------------------------------------------------------

% To have just one problem per page, simply put a \clearpage after each problem
\begin{homeworkProblem}

\begin{enumerate}
\item List names of all the languages in the database (sorted alphabetically)?
\begin{lstlisting}[language=SQL]
SELECT * FROM language ORDER BY name ASC;
	\end{lstlisting}
	\begin{center}
\includegraphics[width=0.75\columnwidth]{1} %

\end{center}

\item List full names of actors with "GER" in their last name, ordered by their first name
\begin{lstlisting}[language=SQL]
SELECT concat(first_name, ' ', last_name) AS full_name
FROM actor
WHERE last_name LIKE '%GER%'
ORDER BY first_name;
	\end{lstlisting}
	\begin{center}
\includegraphics[width=0.75\columnwidth]{2} %
\end{center}


\item Find all the addresses where postal code starts with "57", and return addresses sorted.
\begin{lstlisting}[language=SQL]
SELECT *
FROM address
WHERE postal_code LIKE "57%"
ORDER BY address;
	\end{lstlisting}
	\begin{center}
\includegraphics[width=0.75\columnwidth]{3} %
\end{center}


\item How many films involve a "DWARF" in their titles?
\begin{lstlisting}[language=SQL]
SELECT COUNT(*) FROM film WHERE title LIKE "%DWARF%";
	\end{lstlisting}
	\begin{center}
\includegraphics[width=0.75\columnwidth]{4} %
\end{center}


\item Find full names of actors who played in a film involving ’WAR’ in title and longer than 2.5 hours, along with the title, run length and release year of the movie, sorted by the actors’ last names.
\begin{lstlisting}[language=SQL]
SELECT
  concat(first_name, " ", last_name) AS full_name,
  title,
  release_year
FROM film_actor
  JOIN actor
    ON film_actor.actor_id = actor.actor_id
  JOIN
  (SELECT
     film_id,
     title,
     release_year
   FROM film
   WHERE title LIKE "%WAR%" AND length > 150) AS t1
    ON t1.film_id = film_actor.film_id
ORDER BY actor.last_name;
	\end{lstlisting}
	\begin{center}
\includegraphics[width=0.75\columnwidth]{5} %
\end{center}

\item Find all the film categories in which there are between 55 and 65 films. Return the names of these categories and the number of films per category, sorted by the number of films descending.
\begin{lstlisting}[language=SQL]
SELECT
  tmp.name,
  tmp.count
FROM
  (SELECT
     category.name AS name,
     category.category_id,
     count(*)      AS count
   FROM film_category
     JOIN category ON film_category.category_id = category.category_id
   GROUP BY category_id) AS tmp
WHERE count > 55 AND count < 65
ORDER BY count DESC;
	\end{lstlisting}
	\begin{center}
\includegraphics[width=0.75\columnwidth]{6} %
\end{center}

\newpage

\item In how many film categories is the average di erence between the film replacement cost and the rental rate larger than 17?
\begin{lstlisting}[language=SQL]
SELECT count(*)
FROM
  (SELECT 1
   FROM film
     JOIN film_category
       ON film_category.film_id = film.film_id
   GROUP BY film_category.category_id
   HAVING abs(avg(replacement_cost) - avg(rental_rate)) > 17)
    AS t;
	\end{lstlisting}
	\begin{center}
\includegraphics[width=0.75\columnwidth]{7} %
\end{center}

\item Find the address district(s) name(s) such that the minimal postal code in the district(s) is maximal over all the districts. Make sure your query ignores empty postal codes and district names.
\begin{lstlisting}[language=SQL]
SELECT
  district,
  min(postal_code) AS min_postal_code
FROM address
WHERE postal_code IS NOT NULL
GROUP BY district
ORDER BY min_postal_code DESC
LIMIT 1;
	\end{lstlisting}
	\begin{center}
\includegraphics[width=0.75\columnwidth]{8} %
\end{center}


\item Find the names (first and last) of all the actors and customers whose first name is the same as the first name of the actor with ID 101 (exclude the actor with ID 101).
\begin{lstlisting}[language=SQL]
CREATE VIEW tmp AS
  SELECT first_name
  FROM actor
  WHERE actor_id = 101;

SELECT
  first_name,
  last_name
FROM customer
WHERE first_name = (SELECT first_name
                    FROM tmp)
UNION
SELECT
  first_name,
  last_name
FROM actor
WHERE first_name = (SELECT first_name
                    FROM tmp)
      AND actor_id <> 101;
	\end{lstlisting}
	\begin{center}
\includegraphics[width=0.75\columnwidth]{9} %
\end{center}

\end{enumerate}

\end{homeworkProblem}

\newpage



%----------------------------------------------------------------------------------------

\end{document}