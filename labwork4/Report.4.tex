\documentclass{article}

\usepackage{fancyhdr} % Required for custom headers
\usepackage{lastpage} % Required to determine the last page for the footer
\usepackage{amsmath}
\usepackage{extramarks} % Required for headers and footers
\usepackage[usenames,dvipsnames]{color} % Required for custom colors
\usepackage{graphicx} % Required to insert images
\usepackage{listings} % Required for insertion of code
\usepackage{courier} % Required for the courier font
\usepackage{lipsum} % Used for inserting dummy 'Lorem ipsum' text into the template

% Margins
\topmargin=-0.45in
\evensidemargin=0in
\oddsidemargin=0in
\textwidth=6.5in
\textheight=9.0in
\headsep=0.25in

\linespread{1.1} % Line spacing

% Set up the header and footer
\pagestyle{fancy}
\lhead{\hmwkAuthorName} % Top left header
\chead{\hmwkClass\ (\hmwkClassInstructor\ \hmwkClassTime): \hmwkTitle} % Top center head
\rhead{\firstxmark} % Top right header
\lfoot{\lastxmark} % Bottom left footer
\cfoot{} % Bottom center footer
\rfoot{Page\ \thepage\ of\ \protect\pageref{LastPage}} % Bottom right footer
\renewcommand\headrulewidth{0.4pt} % Size of the header rule
\renewcommand\footrulewidth{0.4pt} % Size of the footer rule

\setlength\parindent{0pt} % Removes all indentation from paragraphs

%----------------------------------------------------------------------------------------
%	CODE INCLUSION CONFIGURATION
%----------------------------------------------------------------------------------------

\definecolor{MyDarkGreen}{rgb}{0.0,0.4,0.0} % This is the color used for comments
\lstloadlanguages{Perl} % Load Perl syntax for listings, for a list of other languages supported see: ftp://ftp.tex.ac.uk/tex-archive/macros/latex/contrib/listings/listings.pdf
\lstset{language=Perl, % Use Perl in this example
        frame=single, % Single frame around code
        basicstyle=\small\ttfamily, % Use small true type font
        keywordstyle=[1]\color{Blue}\bf, % Perl functions bold and blue
        keywordstyle=[2]\color{Purple}, % Perl function arguments purple
        keywordstyle=[3]\color{Blue}\underbar, % Custom functions underlined and blue
        identifierstyle=, % Nothing special about identifiers                                         
        commentstyle=\usefont{T1}{pcr}{m}{sl}\color{MyDarkGreen}\small, % Comments small dark green courier font
        stringstyle=\color{Purple}, % Strings are purple
        showstringspaces=false, % Don't put marks in string spaces
        tabsize=5, % 5 spaces per tab
        %
        % Put standard Perl functions not included in the default language here
        morekeywords={rand},
        %
        % Put Perl function parameters here
        morekeywords=[2]{on, off, interp},
        %
        % Put user defined functions here
        morekeywords=[3]{test},
       	%
        morecomment=[l][\color{Blue}]{...}, % Line continuation (...) like blue comment
        numbers=left, % Line numbers on left
        firstnumber=1, % Line numbers start with line 1
        numberstyle=\tiny\color{Blue}, % Line numbers are blue and small
        stepnumber=5 % Line numbers go in steps of 5
}

% Creates a new command to include a perl script, the first parameter is the filename of the script (without .pl), the second parameter is the caption
\newcommand{\perlscript}[2]{
\begin{itemize}
\item[]\lstinputlisting[caption=#2,label=#1]{#1.pl}
\end{itemize}
}

%----------------------------------------------------------------------------------------
%	DOCUMENT STRUCTURE COMMANDS
%	Skip this unless you know what you're doing
%----------------------------------------------------------------------------------------

% Header and footer for when a page split occurs within a problem environment
\newcommand{\enterProblemHeader}[1]{
\nobreak\extramarks{#1}{#1 continued on next page\ldots}\nobreak
\nobreak\extramarks{#1 (continued)}{#1 continued on next page\ldots}\nobreak
}

% Header and footer for when a page split occurs between problem environments
\newcommand{\exitProblemHeader}[1]{
\nobreak\extramarks{#1 (continued)}{#1 continued on next page\ldots}\nobreak
\nobreak\extramarks{#1}{}\nobreak
}

\setcounter{secnumdepth}{0} % Removes default section numbers
\newcounter{homeworkProblemCounter} % Creates a counter to keep track of the number of problems

\newcommand{\homeworkProblemName}{}
\newenvironment{homeworkProblem}[1][Problem \arabic{homeworkProblemCounter}]{ % Makes a new environment called homeworkProblem which takes 1 argument (custom name) but the default is "Problem #"
\stepcounter{homeworkProblemCounter} % Increase counter for number of problems
\renewcommand{\homeworkProblemName}{#1} % Assign \homeworkProblemName the name of the problem
\section{\homeworkProblemName} % Make a section in the document with the custom problem count
\enterProblemHeader{\homeworkProblemName} % Header and footer within the environment
}{
\exitProblemHeader{\homeworkProblemName} % Header and footer after the environment
}

\newcommand{\problemAnswer}[1]{ % Defines the problem answer command with the content as the only argument
\noindent\framebox[\columnwidth][c]{\begin{minipage}{0.98\columnwidth}#1\end{minipage}} % Makes the box around the problem answer and puts the content inside
}

\newcommand{\homeworkSectionName}{}
\newenvironment{homeworkSection}[1]{ % New environment for sections within homework problems, takes 1 argument - the name of the section
\renewcommand{\homeworkSectionName}{#1} % Assign \homeworkSectionName to the name of the section from the environment argument
\subsection{\homeworkSectionName} % Make a subsection with the custom name of the subsection
\enterProblemHeader{\homeworkProblemName\ [\homeworkSectionName]} % Header and footer within the environment
}{
\enterProblemHeader{\homeworkProblemName} % Header and footer after the environment
}

%----------------------------------------------------------------------------------------
%	NAME AND CLASS SECTION
%----------------------------------------------------------------------------------------

\newcommand{\hmwkTitle}{Report \#3} % Assignment title
\newcommand{\hmwkDueDate}{Tuesday,\ November 7,\ 2017} % Due date
\newcommand{\hmwkClass}{MI\ 1.02} % Course/class
\newcommand{\hmwkClassTime}{5:30pm} % Class/lecture time
\newcommand{\hmwkClassInstructor}{Nghiem Thi Phuong} % Teacher/lecturer
\newcommand{\hmwkAuthorName}{Cao Anh Quan} % Your name

%----------------------------------------------------------------------------------------
%	TITLE PAGE
%----------------------------------------------------------------------------------------

\title{
\vspace{2in}
\textmd{\textbf{\hmwkClass:\ \hmwkTitle}}\\
\normalsize\vspace{0.1in}\small{Due\ on\ \hmwkDueDate}\\
\vspace{0.1in}\large{\textit{\hmwkClassInstructor\ \hmwkClassTime}}
\vspace{3in}
}

\author{\textbf{\hmwkAuthorName}}
\date{} % Insert date here if you want it to appear below your name

%----------------------------------------------------------------------------------------

\begin{document}

\maketitle

%----------------------------------------------------------------------------------------
%	TABLE OF CONTENTS
%----------------------------------------------------------------------------------------

%\setcounter{tocdepth}{1} % Uncomment this line if you don't want subsections listed in the ToC


\newpage

%----------------------------------------------------------------------------------------
%	PROBLEM 1
%----------------------------------------------------------------------------------------

% To have just one problem per page, simply put a \clearpage after each problem
\begin{homeworkProblem}

\begin{enumerate}
	\item All info of all employees 
	\begin{lstlisting}[language=SQL]
INSERT INTO departments VALUES ('d010', 'Research and Development');
INSERT INTO departments VALUES ('d011', 'Marketing and Sales');

UPDATE IGNORE dept_emp
SET dept_no = 'd010'
WHERE dept_no = 'd005' OR dept_no = 'd008';
UPDATE IGNORE dept_emp
SET dept_no = 'd011'
WHERE dept_no = 'd001' OR dept_no = 'd007';

UPDATE IGNORE dept_manager
SET dept_no = 'd010'
WHERE dept_no = 'd005' OR dept_no = 'd008';
UPDATE IGNORE dept_manager
SET dept_no = 'd011'
WHERE dept_no = 'd001' OR dept_no = 'd007';

DELETE FROM departments
WHERE dept_no = 'd005' OR dept_no = 'd008' OR dept_no = 'd001' OR dept_no = 'd007';
	\end{lstlisting}
  \item All info of all departments 
  \begin{lstlisting}[language=SQL]
	SELECT * FROM departments;
	\end{lstlisting}
  \item Full names of all employees 
  \begin{lstlisting}[language=SQL]
	SELECT concat(first_name,' ',last_name) as full_name FROM employees;
	\end{lstlisting}
	
  \item Names of all departments 
  \begin{lstlisting}[language=SQL]
	SELECT dept_name FROM departments;
	\end{lstlisting}
	
	
  \item Full names of employees working in "Sales" department
   \begin{lstlisting}[language=SQL]
	SELECT concat(first_name,' ',last_name) as full_name
		FROM employees
		JOIN dept_emp ON employees.emp_no = dept_emp.emp_no
		JOIN departments ON dept_emp.dept_no = departments.dept_no
		WHERE dept_name = 'Sales';
	\end{lstlisting}
	
	
  \item Full names of male employees working in "Finance" department 
\begin{lstlisting}[language=SQL]
	SELECT concat(first_name,' ',last_name) as full_name
		FROM employees
		JOIN dept_emp ON employees.emp_no = dept_emp.emp_no
		JOIN departments ON dept_emp.dept_no = departments.dept_no
		WHERE gender = 'M' AND dept_name = 'Finance';
	\end{lstlisting}  
  
  \item Salaries of female employees working in "Marketing" department 
  \begin{lstlisting}[language=SQL]
	SELECT employees.emp_no, salary
		FROM employees
		JOIN dept_emp ON employees.emp_no = dept_emp.emp_no
		JOIN departments ON dept_emp.dept_no = departments.dept_no
		JOIN salaries ON employees.emp_no = salaries.emp_no
		WHERE gender = 'F' AND dept_name = 'Marketing';
	\end{lstlisting}  
	
	\newpage
	\item Full names of employees who have the same last name as their manager 
	\begin{lstlisting}[language=SQL]
SELECT emp.full_name
FROM
  (SELECT
     concat(first_name, ' ', last_name) AS full_name,
     last_name,
     employees.emp_no                   AS empno
   FROM employees
     JOIN dept_emp ON employees.emp_no = dept_emp.emp_no) AS emp
  JOIN (SELECT
          last_name,
          employees.emp_no AS empno
        FROM employees
          JOIN dept_manager ON employees.emp_no = dept_manager.emp_no) AS man
    ON emp.last_name = man.last_name;
	\end{lstlisting}  
  
  \item Full names of managers who have been doing the job at least twice (use name g Count()(R) to count) 
\begin{lstlisting}[language=SQL]
SELECT concat(first_name, ' ', last_name) AS full_name
FROM employees
  JOIN
  (SELECT emp_no
   FROM dept_manager
   GROUP BY emp_no
   HAVING count(*) > 1) AS R2 ON R2.emp_no = employees.emp_no;
	\end{lstlisting}  
	  
  
  \item Full names of employees who was paid more than \$100000 
  \begin{lstlisting}[language=SQL]
SELECT concat(first_name, ' ', last_name) AS full_name
FROM employees
  JOIN salaries ON employees.emp_no = salaries.emp_no
WHERE salary > 100000;
	\end{lstlisting}  
  \item Names of all departments that have employees paid more than \$100000
  
  \begin{lstlisting}[language=SQL]
(SELECT dept_name
 FROM departments
   JOIN dept_emp ON departments.dept_no = dept_emp.dept_no
   JOIN salaries ON dept_emp.emp_no = salaries.emp_no
 WHERE salary > 100000)
UNION
(SELECT dept_name
 FROM departments
   JOIN dept_manager ON departments.dept_no = dept_manager.dept_no
   JOIN salaries ON salaries.emp_no = dept_manager.emp_no
 WHERE salary > 100000);
	\end{lstlisting}  
	
\end{enumerate}

\end{homeworkProblem}

\newpage



%----------------------------------------------------------------------------------------

\end{document}